\newpage
\section{Definition of Mass Transfer}

\subsection*{Resources}
\begin{itemize}
    \item Chapter 14, introduction
\end{itemize}

\subsection*{Challenge}
Add the points of the following conditions which constitute diffusive mass-transfer

1 point: Evaporation of water vapour into the air

2 points: Water being pumped through a pipe

4 points: Dissolving of sugar into tea

8 points: Aeration of waste-water

16 points: Motion of air around a room due to the presence of a fan

\subsection*{Solution}
(Enter as an integer)

\hash{a}{e3d829}




\newpage
\section{Diffusion in the long time limit}

\subsection*{Resources}
\begin{itemize}
    \item Book: 14.1.1 to 14.1.2
    \item Video: \url{https://www.youtube.com/watch?v=-FLv0uxLrDI}
\end{itemize}

\subsection*{Challenge}
Consider a box of volume \SI{1}{\cubic\meter}. The box contains 1 mole of gas. At time $t=0$, all the gas molecules in the left 1/4 of the box are labeled $A$. As time goes to $t=\infty$, what will the density of the molecules labeled $A$ be in the right half of the box?

\subsection*{Solution}
Units: Moles / $m^3$

\hash{b}{f5f814} 




\newpage
\section{Definitions of quantities I}

\subsection*{Resources}
\begin{itemize}
    \item Book: 14.1.1 to 14.1.2
    \item Video: \url{https://www.youtube.com/watch?v=-FLv0uxLrDI}
\end{itemize}

\subsection*{Challenge}
Assuming air is made up exclusively of oxygen and nitrogen in the ratio 0.21:0.79, what are their mass-fractions?

\subsection*{Solutions}
Oxygen: 0.233 \\
Nitrogen: 0.767




\newpage
\section{Definitions of quantities II}

\subsection*{Resources}
\begin{itemize}
    \item Book: 14.1.1 to 14.1.2
    \item Video: \url{https://www.youtube.com/watch?v=-FLv0uxLrDI}
\end{itemize}

\subsection*{Challenge}
Japan imports substantial amounts of LNG which is a mixture of the following gases:

\begin{tabular}[c]{|c|c|}
    \hline
    \textbf{Liquid} & \textbf{Mol \%}\\
    \hline
    Methane         & 93.5  \\
    Ethane          & 4.6   \\
    Propane         & 1.2   \\
    Carbon dioxide  & 0.7   \\
    \hline
\end{tabular}

The masses of Methane, Ethane, Propane and Carbon Dioxide are 16, 30, 44 and 44 g/mol respectively.

Assuming ideal gases, calculate the following:

1. The mole-fraction of ethane

2. The mass-fraction of ethane

3. The average molecular mass of the mixture

4. The mass-density of the gas when heated to \SI{207}{\kelvin} under a total pressure of \SI{1.4e5}{\pascal}

5. The partial pressure of the methane when the total pressure is \SI{1.4e5}{\pascal}

\subsection*{Solutions}

1. (enter as a decimal to 3 decimal places) \hash{c}{da59a9}

2. (enter as a decimal to 3 decimal places) \hash{d}{15ff4e}

3. (enter as a decimal to 3 decimal places in units of g/mol) \hash{e}{234c45}

4. \SI{1397}{\kg\per\cubic\meter}

5. (enter as an integer in units of kPa) \hash{f}{e9e087}




\newpage
\section{Mass diffusivity}

\subsection*{Resources}
\begin{itemize}
    \item Book: 14.1.3 - 14.1.4, Table A-8
\end{itemize}

\subsection*{Challenge}
Estimate the mass diffusivity of the following gases in air at 350 K and 1 atm pressure:

1. Ammonia\\
2. Hydrogen

\subsection*{Solutions}
1. \SI{0.36e-4}{\square\meter\per\second}\\
2. \SI{0.52e-4}{\square\meter\per\second}




\newpage
\section{Cases of diffusion}

\subsection*{Resources}
\begin{itemize}
    \item Book: 14.1.3 - 14.1.4
\end{itemize}

\subsection*{Challenge}
Considering air in a closed, cylindrical container with its axis vertical and with opposite ends maintained at different temperatures. Assume the total pressure of the air is uniform throughout the container.

Consider each of the following conditions:

1. The bottom surface is colder than the top surface\\
2. The top surface is colder than the bottom surface

For each condition, write a few sentences explaining a) if there is any motion of the air and b) if mass transfer occurs.

\subsection*{Solutions}
Please compare your answer with your partner.




\newpage
\section{Diffusion coefficient equivalency}

\subsection*{Resources}
\begin{itemize}
    \item Video: \url{https://www.youtube.com/watch?v=NTlR18NyqAE}
\end{itemize}

\subsection*{Challenge}
Prove that in a binary mixture, the diffusion coefficient of gas ``A'' in ``B'' is the same as the diffusion coefficient of gas ``B'' in ``A'' (ie, $D_{AB} = D_{BA}$).
